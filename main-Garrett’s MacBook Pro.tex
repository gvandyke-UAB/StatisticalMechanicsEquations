%%%%%%%%%%%%%%%%%%%%%%%%%%%%%%%%%%%%%%%%%
% Lachaise Assignment
% LaTeX Template
% Version 1.0 (26/6/2018)
%
% This template originates from:
% http://www.LaTeXTemplates.com
%
% Authors:
% Marion Lachaise & François Févotte
% Vel (vel@LaTeXTemplates.com)
%
% License:
% CC BY-NC-SA 3.0 (http://creativecommons.org/licenses/by-nc-sa/3.0/)
% 
%%%%%%%%%%%%%%%%%%%%%%%%%%%%%%%%%%%%%%%%%

%----------------------------------------------------------------------------------------
%	PACKAGES AND OTHER DOCUMENT CONFIGURATIONS
%----------------------------------------------------------------------------------------

\documentclass{article}
\usepackage{physics}

\input{structure.tex} % Include the file specifying the document structure and custom commands

%----------------------------------------------------------------------------------------
%	ASSIGNMENT INFORMATION
%----------------------------------------------------------------------------------------

\title{Stat Mech Bible} % Title of the assignment

\author{Garrett Van Dyke\\ \texttt{gvandyke@uab.edu}} % Author name and email address

\date{University of Alabama at Birmingham --- \today} % University, school and/or department name(s) and a date

%----------------------------------------------------------------------------------------

\begin{document}

\maketitle % Print the title

%----------------------------------------------------------------------------------------
%	INTRODUCTION
%----------------------------------------------------------------------------------------

\section*{Formulas} % Unnumbered section

This is my attempt to catalog everything not included on the Statistical Mechanics qualifying exam equation sheets. Here we go.
% Math equation/formula
\begin{equation}
	z=\sum_n e^{-\beta H_n} \text{    For when the phase space is quantized}
\end{equation}

\begin{equation}
	z=\idotsint_{p_N, q_N} e^{-\beta H_N} \frac{dp_N dq_N}{\hbar^{3N}} \text{   For a continuous phase space}
\end{equation}

\begin{equation}
	P(E_i)=\frac{1}{z}e^{-\beta E_i}
\end{equation}

\begin{equation}
	g(E)=\sum_i \delta(E-E_i) \text{   For discrete energies}
\end{equation}

\begin{equation}
	z=\int g(E)e^{-\beta E}dE
\end{equation}

\begin{equation}
	F=-kT\ln z
\end{equation}

\begin{equation}
	U=-\pdv{\beta}\ln(z)
\end{equation}

\begin{equation}
	S=k\ln\Omega(E)=-\pdv{F}{T}
\end{equation}

\begin{equation}
	\rho=-\pdv{F}{V}
\end{equation}

\begin{equation}
	\mu=\pdv{F}{N}
\end{equation}

\begin{equation}
	M=-kT\pdv{B}\ln z
\end{equation}

\begin{equation}
	\chi=\pdv{M}{B}
\end{equation}



\end{document}
