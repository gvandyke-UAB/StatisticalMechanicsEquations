%%%%%%%%%%%%%%%%%%%%%%%%%%%%%%%%%%%%%%%%%
% Lachaise Assignment
% LaTeX Template
% Version 1.0 (26/6/2018)
%
% This template originates from:
% http://www.LaTeXTemplates.com
%
% Authors:
% Marion Lachaise & François Févotte
% Vel (vel@LaTeXTemplates.com)
%
% License:
% CC BY-NC-SA 3.0 (http://creativecommons.org/licenses/by-nc-sa/3.0/)
% 
%%%%%%%%%%%%%%%%%%%%%%%%%%%%%%%%%%%%%%%%%

%----------------------------------------------------------------------------------------
%	PACKAGES AND OTHER DOCUMENT CONFIGURATIONS
%----------------------------------------------------------------------------------------

\documentclass{article}
\usepackage{physics}
\usepackage{commath}
\usepackage{datetime}
\settimeformat{ampmtime}

\input{structure.tex} % Include the file specifying the document structure and custom commands

%----------------------------------------------------------------------------------------
%	ASSIGNMENT INFORMATION
%----------------------------------------------------------------------------------------

\title{Stat Mech Bible} % Title of the assignment

\author{Garrett Van Dyke\\ \texttt{gvandyke@uab.edu}} % Author name and email address

\date{UAB --- Last compiled: \today\ at \currenttime} % University, school and/or department name(s) and a date

%----------------------------------------------------------------------------------------

\begin{document}

\maketitle % Print the title

%----------------------------------------------------------------------------------------
%	INTRODUCTION
%----------------------------------------------------------------------------------------

 % Unnumbered section

This is my attempt to catalog everything not included on the Statistical Mechanics qualifying exam equation sheets. Here we go. \medskip

\textbf{Let it be hereby known now and forevermore}: $z$ will denote fugacity, $Z$ will denote the canonical partition function, and $\mathcal{Z}$ will denote the Grand Canonical partition function. The qual equation sheet does it differently ($z$ for fugacity, $\mathcal{Z}$ for the partition function, and $\mathcal{Q}$ for the Grand Canonical partition function), but this is written to match in-class notes.\medskip

\textbf{Let it also hereby be known now and forevermore}: Don't forget to add a factor of $\frac{1}{N!}$ on partition functions when your particles are indistinguishable.

\section{General Math}

\begin{equation}
	\dif f(x,y)= \left(\pdv{f}{x}\right)\dif x + \left(\pdv{f}{y}\right)\dif y
\end{equation}

\begin{equation}
	\left(\pdv{x}{y}\right)_z\left(\pdv{y}{z}\right)_x+\left(\pdv{x}{z}\right)_y=0
\end{equation}

\begin{equation}
	\left\langle a \right\rangle=\int a \, P_a \dif a \; \textbf{ OR } \; \left\langle a \right\rangle=\sum_a a \, P_a
\end{equation}


\section{Formulas}
\subsection{Canonical Ensemble: \normalfont Exchange of energy}
\subsubsection{Specific to Ideal Gas}
\begin{equation}
	\Omega(E)=\frac{1}{N!}\alpha^N \frac{E}{N}V^N
\end{equation}

\begin{equation}
	Z_N=\left(\frac{V}{\lambda^3}\right)^N
\end{equation}
\subsubsection{In General}
\begin{equation}
	\Omega(E): \text{   Number of states for a given energy}
\end{equation}

\begin{equation}
	Z=\sum_n e^{-\beta H_n}: \text{   For a quantized phase space}
\end{equation}

\begin{equation}
	Z=\idotsint_{p_N, q_N} e^{-\beta H_N} \frac{\dif p_N \dif q_N}{h^{3N}}: \text{   For a continuous phase space}
\end{equation}

\begin{equation}
	P(E_i)=\frac{1}{Z}e^{-\beta E_i}
\end{equation}

\begin{equation}
	g(E)=\sum_i \delta(E-E_i): \text{   For discrete energies}
\end{equation}

\begin{equation}
	Z=\int g(E) \, e^{-\beta E}\dif E
\end{equation}

\begin{equation}
	F=-kT\ln Z
\end{equation}

\begin{equation}
	U=-\pdv{\beta}\ln Z
\end{equation}

\begin{equation}
	S=-k\sum_i \rho_i \ln \rho_i=-k\sum_i P_i \ln P_i=k\ln\Omega(E)=-\pdv{F}{T}
\end{equation}

\begin{equation}
	T=\left(\pdv{U}{S}\right)_{V, N}
\end{equation}

\begin{equation}
	p=-\pdv{F}{V}
\end{equation}

\begin{equation}
	\mu=\pdv{F}{N}
\end{equation}

\begin{equation}
	M=-kT\pdv{B}\ln Z
\end{equation}

\begin{equation}
	\chi=\pdv{M}{B}\approx\frac{C}{T}:\text{   where C is the Curie constant}
\end{equation}

\begin{equation}
	c_V=\pdv{U}{T}=k\beta ^2\pdv[2]{\beta}\ln Z
\end{equation}

\subsection{Grand Canonical Ensemble: \normalfont Exchange of energy and number of particles}

\begin{equation}
	P(U,N)=\frac{1}{\mathcal{Z}}e^{-\beta(U-\mu N})
\end{equation}

\begin{equation}
	\mathcal{Z}=\sum_U \sum_N e^{-\beta(U-\mu N})=\sum_N e^{\beta\mu N}Z_N=\sum_N Z_Nz^N
\end{equation}

\begin{equation}
	\Phi_G=-kT\ln\mathcal{Z}
\end{equation}

\begin{equation}
	U=\left.-\pdv{\beta}\ln\mathcal{Z}\right\rvert_z
\end{equation}

\begin{equation}
	N=z\pdv{z}\ln\mathcal{Z}=-\pdv{\Phi_G}{\mu}
\end{equation}

\begin{equation}
	p=\frac{\Phi_G}{V}
\end{equation}

\section{Energies}

Thermodynamic Potential:
\begin{equation}
	U(S, V, N)
\end{equation}
%
Helmholtz Free Energy:
\begin{equation}
	F(T, V, N)=U-TS
\end{equation}
%
Gibbs Free Energy:
\begin{equation}
	G(T, p, N)=U-TS-pV
\end{equation}
%
Grand Potential:
\begin{equation}
	\Phi_G(T, V, \mu)=U-TS-\mu N
\end{equation}

\end{document}
